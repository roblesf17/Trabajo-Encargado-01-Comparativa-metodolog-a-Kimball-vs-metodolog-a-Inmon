%
\documentclass[%
 reprint,
 amsmath,amssymb,
 aps,
]{revtex4-1}

\usepackage{graphicx}% Include figure files
\usepackage{dcolumn}% Align table columns on decimal point
\usepackage{bm}% bold math


\begin{document}


\title{Comparativa Metodologia Kimball vs Metodología Inmon}
\author{Robles Flores, Anthony Richard	               (2016056192)}
\author{Estrella Palacios, Katherine Lizbeth			(2015050948)}
\author{Sosa Bedoya, Sharon Fiorela					(2016054460)}
\author{Torres Beltran , Johanna Andrea				(2020067849)}

		
\affiliation{%
 Universidad Privada de Tacna \textbackslash Facultad de Ingenieria \textbackslash Escuela Profesional de Ingenieria de Sistemas
}%

\begin{abstract}
\begin{center}
\textbf{Resumen}
\end{center}
Los almacenes de datos (data warehouses en inglés) toman cada día mayor importancia, a medida que las organizaciones pasan de esquemas de sólo recolección de datos a esquemas de análisis de los mismos. En este breve artículo se  tratará de brindar una explicación general de algunas metodologías, en este caso serán la metodología Kimball y la metodología Inmon.
\\

\begin{center}
\textbf{Abstract}
\end{center}
Data warehouses (data warehouses in English) are becoming increasingly important, as organizations move from data-only schemes to data analysis schemes. In this short article we will try to provide a general explanation of some methodologies, in this case they will be the Kimball methodology and the Inmon methodology.
\\
\end{abstract}



\maketitle

%\tableofcontents

\section {Introducción}\label{sec:1}

xxxxxxxxxxxxxxxxxxxxxxxxxxxxxxxxxxxxxx
%-----------------------------------------------------------------
\section{Objetivos}\label{sec:2}
\subsection{General:}
xxxxxxxxxxxxxxxxxxxxxxxxxxxxxxxxxxxxxx
\subsection{Específicos:}
xxxxxxxxxxxxxxxxxxxxxxxxxxxxxxxxxxxxxx

%-----------------------------------------------------------------
\section {Marco Teórico}

\subsection{METODOLOGIA SEGUN INMON}	
xxxxxxxxxxxxxxxxxxxxxxxxxxxxxxxxxxxxxx 


%-------------------------------------------------

\subsection{IMPLEMENTACIONES DE  INMON}
xxxxxxxxxxxxxxxxxxxxxxxxxxxxxxxxxxxxxx 

%-------------------------------------------------
\subsection{ARQUITECTURA SEGUN INMON}
xxxxxxxxxxxxxxxxxxxxxxxxxxxxxxxxxxxxxx 

%-------------------------------------------------
\subsection{METODOLOGIA SEGUN KIMBALL}	
xxxxxxxxxxxxxxxxxxxxxxxxxxxxxxxxxxxxxx 

%-------------------------------------------------
\subsection{IMPLEMENTACIONES DE KIMBALL}
xxxxxxxxxxxxxxxxxxxxxxxxxxxxxxxxxxxxxx 

%-------------------------------------------------
\subsection{ARQUITECTURA SEGUN KIMBALL}

xxxxxxxxxxxxxxxxxxxxxxxxxxxxxxxxxxxxxx 

%-------------------------------------------------

\subsection{COMPARACION DE METODOLOGIAS}	
xxxxxxxxxxxxxxxxxxxxxxxxxxxxxxxxxxxxxx \cite{kumar}\\

%-----------------------------------------------------------------
\section{Análisis}

\begin{itemize}
\item xxxxxxxxxxxxxxxxxxxxxxxxxxxxxxxxxxxxxx 
\item xxxxxxxxxxxxxxxxxxxxxxxxxxxxxxxxxxxxxx 
\end{itemize}
%-----------------------------------------------------------------
\section{Conclusiones}

\begin{itemize}
\item xxxxxxxxxxxxxxxxxxxxxxxxxxxxxxxxxxxxxx 
\item xxxxxxxxxxxxxxxxxxxxxxxxxxxxxxxxxxxxxx 

\end{itemize}


% Bibliografia.
%-----------------------------------------------------------------

\bibliographystyle{apalike}

\bibliography{Bibliografia}

\end{document}
