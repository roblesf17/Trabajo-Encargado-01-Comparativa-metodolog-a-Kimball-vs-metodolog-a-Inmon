%
\documentclass[%
 reprint,
 amsmath,amssymb,
 aps,
]{revtex4-1}

\usepackage{graphicx}% Include figure files
\usepackage{dcolumn}% Align table columns on decimal point
\usepackage{bm}% bold math


\begin{document}


\title{Comparativa Metodologia Kimball vs Metodología Inmon}
\author{Robles Flores, Anthony Richard	               (2016056192)}
\author{Estrella Palacios, Katherine Lizbeth			(2015050948)}
\author{Sosa Bedoya, Sharon Fiorela					(2016054460)}
\author{Torres Beltran , Johanna Andrea				(2020067849)}

		
\affiliation{%
 Universidad Privada de Tacna \textbackslash Facultad de Ingenieria \textbackslash Escuela Profesional de Ingenieria de Sistemas
}%

\begin{abstract}
\begin{center}
\textbf{Resumen}
\end{center}
Los almacenes de datos (data warehouses en inglés) toman cada día mayor importancia, a medida que las organizaciones pasan de esquemas de sólo recolección de datos a esquemas de análisis de los mismos. En este breve artículo se  tratará de brindar una explicación general de algunas metodologías, en este caso serán la metodología Kimball y la metodología Inmon.
\\

\begin{center}
\textbf{Abstract}
\end{center}
Data warehouses (data warehouses in English) are becoming increasingly important, as organizations move from data-only schemes to data analysis schemes. In this short article we will try to provide a general explanation of some methodologies, in this case they will be the Kimball methodology and the Inmon methodology.
\\
\end{abstract}



\maketitle

%\tableofcontents

\section {Introducción}\label{sec:1}

Actualmente las organizaciones utilizan la información y el conocimiento para apoyar la toma de sus decisiones estratégicas, y de este modo lograr sus metas y mejorar sus procesos.
Uno de los desafíos que enfrentan hoy las organizaciones, es el aumento de datos, lo que ha generado dos grandes problemas; el primero, identificar los datos relevantes para dar seguimiento a su estrategia organizacional, y lograr que se cumplan los planes con las metas establecidas.
 Y el segundo problema, la capacidad para administrar esta gran cantidad de datos.
Un almacén de datos  según Inmon, es una colección de datos orientada a un determinado ámbito (empresa, organización, etc.), integrado, no volatil y variable en el tiempo, que ayuda a la toma de decisiones en la entidad en la que se utiliza. 
Se trata, sobre todo, de un historial completo de la organización, mas alla de la informacion transaccional y operacional, almacenado en una base de datos diseñada para favorecer el análisis y la divulgación eficiente de datos (especialmente con herramientas OLAP, de procesamiento analítico en línea). Por otra parte Kimball la define como una copia de los datos transaccionales estructurados específicamente para consultas y análisis. 
En este breve artículo intentaremos brindar una explicación general de dos de las metodologías más usadas, la metodología de Kimball y metodología Inmon. \cite{estrella1}
%-----------------------------------------------------------------
\section{Objetivos}\label{sec:2}
\subsection{General:}
Dar una visión clara de BI, desde las perspectivas de los autores que sentaron las bases que son Ralph Kimball y Bill Inmon, para la mejora de las estrategias del negocio al que se desee implementar las herramientas de BI.
\subsection{Específicos:}
 Describir las metodologías propuestas por los principales autores de BI desde las perspectivas de sus creadores Ralph Kimball y Bill Inmon.


%-----------------------------------------------------------------
\section {Marco Teórico}

\subsection{METODOLOGIA SEGUN INMON}	
Un almacén de datos (data warehouse, DW) según Inmon es una colección de datos orientada a un determinado ámbito (empresa, organización, etc.), integrado, no volátil y variable en el tiempo, que ayuda a la toma de decisiones en la entidad en la que se utiliza. Se trata, sobre todo, de un historial completo de la organización, más allá de la información transaccional y operacional, almacenado en una base de datos diseñada para favorecer el análisis y la divulgación eficiente de datos (especialmente con herramientas OLAP, de procesamiento analítico en línea). \cite{estrella2}

Consta de las siguientes características:
\begin{itemize}
		\item \textbf{Orientado a temas:} Los datos en la base de datos están organizados de manera que todos los elementos de datos relativos al mismo evento u objeto del mundo real queden unidos entre sí.
		\item \textbf{Integrado:} La base de datos contiene los datos de todos los sistemas operacionales de la organización, y dichos datos deben ser consistentes. 
		\item \textbf{No volátil:} La información no se modifica ni se elimina, una vez almacenado un dato, éste se convierte en información de sólo lectura, y se 	mantiene para futuras consultas.
		\item \textbf{Variante en el tiempo:}Los cambios producidos en los datos a lo largo del tiempo quedan registrados para que los informes que se puedan generar reflejen esas variaciones.

\end{itemize}


El enfoque Inmon tambien se referencia normalmente como Top-down. Los datos son extraidos de los sistemas operacionales por los procesos ETL y cargados en las areas de stage, donde son validados y consolidados en el DW corporativo, donde ademas existen los llamados metadatos que documentan de una forma clara y precisa el contenido del DW. Una vez realizado este proceso, los procesos de refresco de los Data Mart departamentales obtienen la información de el, y con las consiguientes transformaciones, organizan los datos en las estructuras particulares requeridas por cada uno de ellos, refrescando su contenido.


%-------------------------------------------------

\subsection{IMPLEMENTACIONES DE  INMON}
xxxxxxxxxxxxxxxxxxxxxxxxxxxxxxxxxxxxxx 

%-------------------------------------------------
\subsection{ARQUITECTURA SEGUN INMON}
xxxxxxxxxxxxxxxxxxxxxxxxxxxxxxxxxxxxxx 

%-------------------------------------------------
\subsection{METODOLOGIA SEGUN KIMBALL}	
xxxxxxxxxxxxxxxxxxxxxxxxxxxxxxxxxxxxxx 

%-------------------------------------------------
\subsection{IMPLEMENTACIONES DE KIMBALL}
xxxxxxxxxxxxxxxxxxxxxxxxxxxxxxxxxxxxxx 

%-------------------------------------------------
\subsection{ARQUITECTURA SEGUN KIMBALL}

xxxxxxxxxxxxxxxxxxxxxxxxxxxxxxxxxxxxxx 

%-------------------------------------------------

\subsection{COMPARACION DE METODOLOGIAS}	
xxxxxxxxxxxxxxxxxxxxxxxxxxxxxxxxxxxxxx

%-----------------------------------------------------------------
\section{Análisis}

\begin{itemize}
\item xxxxxxxxxxxxxxxxxxxxxxxxxxxxxxxxxxxxxx 
\item xxxxxxxxxxxxxxxxxxxxxxxxxxxxxxxxxxxxxx 
\end{itemize}
%-----------------------------------------------------------------
\section{Conclusiones}

\begin{itemize}
\item xxxxxxxxxxxxxxxxxxxxxxxxxxxxxxxxxxxxxx 
\item xxxxxxxxxxxxxxxxxxxxxxxxxxxxxxxxxxxxxx 

\end{itemize}


% Bibliografia.
%-----------------------------------------------------------------

\bibliographystyle{apalike}

\bibliography{Bibliografia}

\end{document}
